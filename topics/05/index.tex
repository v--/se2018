% arara: pdflatex: { shell: true, interaction: nonstopmode }
% arara: biber
% arara: pdflatex: { shell: true }

\documentclass[numbers=endperiod, DIV=15, bibliography=totocnumbered]{scrartcl}

% Base packages
\usepackage[T2A]{fontenc}
\usepackage[utf8]{inputenc}
\usepackage[bulgarian]{babel}
\usepackage[pdfencoding=unicode]{hyperref}
\usepackage{biblatex}
\usepackage[style=german]{csquotes}

% Base math packages
\usepackage{amsmath}
\usepackage{amssymb}
\usepackage{amsthm}
\usepackage{mathtools}

% Misc packages
\usepackage{ulem} % Line-breaking underlines

% Custom packages
\usepackage{../../common/macros}
\usepackage{../../common/theorems}

% Bibliography
\addbibresource{./references.bib}

% Document
\title{[WIP] Тема 5}
\subtitle{Теореми за средните стойности (Рол, Лагранж и Коши). Формула на Тейлър.}
\author{Янис Василев, \Email{ianis@ivasilev.net}}
\date{6 юли 2019}

\begin{document}

\maketitle

\section{Теория}

Теорията е разписана без външни източници.

\subsection{Анотация}

Изложената анотацията е взета от конспекта~\cite{Syllabus} за 2018г.

\begin{enumerate}
  \item Теоремата на Рол с доказателство, базирано на теоремата на Вайерщрас
  \item Теоремите за средните стойности на Лагранж и Коши
  \item Формула на Тейлър с остатъчен член във формата на Лагранж и Коши
\end{enumerate}

\subsection{Помощни теореми}

\begin{definition}
  Нека $D \subseteq \R$ и $f: D \to \R$ е произволна функция.

  \begin{enumerate}
    \item Казваме, че \uline{функцията $f(x)$ има локален максимум в точка $a \in D$}, ако за някаква околност $U_a$ на $a$ е изпълнено $f(a) \geq f(u)~\forall u \in U_a$.

    \item Казваме, че \uline{функцията $f(x)$ има локален минимум в точка $a \in D$}, ако за някаква околност $U_a$ на $a$ е изпълнено $f(a) \leq f(u)~\forall u \in U_a$.

    \item Ако $f(x)$ има или локален максимум, или локален минимум в точка $a$, казваме, че \uline{$f(x)$ има локален екстремум в тока $a$}.
  \end{enumerate}
\end{definition}

\begin{theorem}[Вайерщрас]
  Нека $a < b$ и функцията $f: [a, b] \to \R$ е непрекъсната. Тогава $f(x)$ достига в $[a, b]$ своите минимум и максимум.
\end{theorem}
\begin{proof}
  Нека $M \coloneqq \sup_{x \in [a, b]} f(x)$. Тогава съществува редица $\{ a_k \} \subseteq [a, b]$, за която съответната редица от функционални стойности $\{ f(a_k) \}$ клони към $M$.

  Тъй като интервалът $[a, b]$ е ограничен, редицата също $\{ a_k \}$ е ограничена и според теоремата на Болцано-Вайерщрас, тя има сходяща подредица.

  Нека $\{ a_{k_i} \}$ е една сходяща подредица на $\{ a_k \}$. Тъй като интервалът $[a, b]$ е затворен, границата $\alpha$ на редицата $\{ a_{k_i} \}$ лежи в този интервал.

  Понеже $f(x)$ е непрекъсната, имаме
  \begin{displaymath}
    \lim_{i \to \infty} f(a_{k_i}) = f(\alpha) = M,
  \end{displaymath}
  където последното равенство получихме от еднозначността на сходимостта на редици.
\end{proof}

\begin{theorem}[Ферма]
  Нека $D \subseteq \R$ и функцията $f: D \to \R$ е диференцируема в точката $a \in D$. Необходимо условия за това $f(x)$ да има локален екстремум в $a$ е производната $f'(a)$ да се анулира.
\end{theorem}
\begin{proof}
  Нека без ограничение на общността $f(x)$ да има локален минимум в $a$. Тогава
  \begin{displaymath}
    f'(a) = \lim_{h \downarrow 0} \frac {f(a+h) - f(a)} h \geq 0,
  \end{displaymath}
  тъй като $f(a+h) \geq f(a)$ за достатъчно малко $h > 0$. Обратно,
  \begin{displaymath}
    f'(a) = \lim_{h \uparrow 0} \frac {f(a+h) - f(a)} h \leq 0,
  \end{displaymath}
  тъй като $f(a+h) \geq f(a)$ за достатъчно малко по абсолютна стойност $h < 0$.

  Заключаваме, че $f'(a) = 0$.
\end{proof}

\subsection{Теореми за средните стойности}

\begin{theorem}[Рол]
  Нека $a < b$, а $f: [a, b] \to \R$ е непрекъсната в $[a, b]$ и диференцируема в $(a, b)$. Ако $f(a) = f(b)$, съществува точка $c \in (a, b)$, такава че $f'(c) = 0$.
\end{theorem}
\begin{proof}
  Разглеждаме три случая:
  \begin{enumerate}
    \item Нека $f(x) \equiv f(a)$ е тъждествено константа. Тогава избираме $c$ да бъде произволна точка от $(a, b)$, тъй като $f'(x) = 0$ за всяко $x \in [a, b]$.
    \item Нека $f(x) > f(a)$ за някое $x$. Според теоремата на Вайерщрас, функцията $f(x)$ достига своя максимум $M_f$ в интервала $[a, b]$ и според допускането ни имаме $M_f > f(a)$. При това този максимум се достига непременно във вътрешността на интервала. Нека $c \in (a, b)$ е точка, за която $f(c) = M_f$. Тъй като $M_f$ е локален екстремум, по теоремата на Ферма имаме $f'(c) = 0$.
    \item Нека $f(x)$ не е тъждествено константа и $f(x) \leq f(a)~\forall x \in [a, b]$. Тогава прилагаме вече доказания случай към функцията $-f(x)$ и получаваме константа $c \in (a, b)$, за която $f'(c) = -f'(c) = 0$.
  \end{enumerate}
\end{proof}

\begin{theorem}[Теорема на Лагранж за крайните нараствания]
  Нека $a < b$, а $f: [a, b] \to \R$ е непрекъсната в $[a, b]$ и диференцируема в $(a, b)$. Тогава съществува точка $c \in (a, b)$, такава че
  \begin{displaymath}
    f'(c) = \frac {f(b) - f(a)} {b-a}.
  \end{displaymath}
\end{theorem}
\begin{proof}
  Разглеждаме спомагателната функция
  \begin{displaymath}
    g(x) \coloneqq f(x) - \frac {f(b) - f(a)} {b-a} x,
  \end{displaymath}
  за която имаме
  \begin{displaymath}
    g(a)
    =
    f(a) - \frac {f(b) - f(a)} {b-a} a
    =
    \frac {b \cdot f(a) - a \cdot f(b)} {b-a}
    =
    f(b) - \frac {f(b) - f(a)} {b-a} b
    =
    g(b).
  \end{displaymath}

  Освен това $g(x)$ е непрекъсната в $[a, b]$ като сума на непрекъснати в $[a, b]$ функции и диференцируема в $(a, b)$ като сума на диференцируеми в $(a, b)$ функции. Това ни позволява да приложим теоремата на Рол към $g(x)$, за да получим константа $c$, за която

  \begin{displaymath}
    0 = g'(c) = f'(c) - \frac {f(b) - f(a)} {b-a}.
  \end{displaymath}

  Но тогава
  \begin{displaymath}
    f'(c) = \frac {f(b) - f(a)} {b-a}.
  \end{displaymath}
\end{proof}

\begin{theorem}[Теорема на Коши]
  Нека $a < b$, а $f: [a, b] \to \R$ и $g: [a, b] \to \R$ са непрекъснати в $[a, b]$ и диференцируеми в $(a, b)$ и нека $g'(x) \neq 0$ за $x \in (a, b)$. Тогава съществува точка $c \in (a, b)$, такава че
  \begin{displaymath}
    \frac {f'(c)} {g'(c)} = \frac {f(b) - f(a)} {g(b) - g(a)}.
  \end{displaymath}
\end{theorem}
\begin{proof}
  Да отбележим първо, че $g'(x) \neq 0$ в $(a, b)$ влече, че условията на теоремата на Рол не са изпълнена за $g(x)$ и следователно $g(a) \neq g(b)$.

  Разглеждаме спомагателната функция
  \begin{displaymath}
    h(x) \coloneqq f(x) - \frac {f(b) - f(a)} {g(b) - g(a)} g(x),
  \end{displaymath}
  за която имаме
  \begin{displaymath}
    h(a)
    =
    f(a) - \frac {f(b) - f(a)} {g(b) - g(a)} g(a)
    =
    \frac {f(a) \cdot g(b) - f(b) \cdot g(a)} {g(b)-g(a)}
    =
    f(b) - \frac {f(b) - f(a)} {g(b) - g(a)} g(b)
    =
    h(b).
  \end{displaymath}

  Функцията $h(x)$ е непрекъсната в $[a, b]$ като линейна комбинация на непрекъснати в $[a, b]$ функции и диференцируема в $(a, b)$ като линейна комбинация на диференцируеми в $(a, b)$ функции. Това ни позволява да приложим теоремата на Рол към $h(x)$, за да получим константа $c$, за която

  \begin{displaymath}
    0 = h'(c) = f'(c) - \frac {f(b) - f(a)} {g(b) - g(a)} g'(c).
  \end{displaymath}

  Но тогава
  \begin{displaymath}
    \frac {f'(c)} {g'(c)} = \frac {f(b) - f(a)} {g(b) - g(a)}.
  \end{displaymath}
\end{proof}

\printbibliography

\end{document}
