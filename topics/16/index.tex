% arara: pdflatex: { shell: true, interaction: nonstopmode }
% arara: biber
% arara: pdflatex: { shell: true }

\documentclass[numbers=endperiod, DIV=15, bibliography=totocnumbered]{scrartcl}

% Base packages
\usepackage[T2A]{fontenc}
\usepackage[utf8]{inputenc}
\usepackage[bulgarian]{babel}
\usepackage[pdfencoding=unicode]{hyperref}
\usepackage{biblatex}
\usepackage[style=german]{csquotes}

% Base math packages
\usepackage{amsmath}
\usepackage{amssymb}
\usepackage{amsthm}
\usepackage{mathtools}

% Misc packages
\usepackage{enumitem} % Customization of enum counters
\usepackage{ulem} % Line-breaking underlines

% Custom packages
\usepackage{../../common/macros}
\usepackage{../../common/theorems}

% Bibliography
\addbibresource{./references.bib}

% Document
\title{Тема 16}
\subtitle{Случайни величини с непрекъснати разпределения. Нормално разпределение. Равномерно разпределение, експоненциално разпределение или гама разпределение. Задачи, в които възникват.}
\author{Янис Василев, \Email{ianis@ivasilev.net}}
\date{21 юни 2019}

\begin{document}

\maketitle

\section{Анотация}

Изложената анотацията е взета от конспекта~\cite{Syllabus} за 2018г.

\subsection{Теория}

\begin{enumerate}
  \item Дефиниция на непрекъснато разпределение на случайна величина
  \item Вероятностна плътност и свойствата ѝ - неотрицателност и нормираност
  \item Дефиниция на моментите на непрекъсната случайна величина
  \item Дефиниция и свойства (без доказателства) на пораждаща моментите/характеристична функция (по избор)
  \item Дефиниция, коректност, мотивиращ пример, пораждаща моментите/характеристична функция, очакване и дисперсия на нормално разпределение и още едно избрано от комисията непрекъснато разпределение
\end{enumerate}

\subsection{Задачи}

Не е даден списък с възможни задачи, затова съм включил разни задачи, давани на държавен изпит.

\section{Теория}

Теорията е представена с минимални препратки към теорията на мярката и е базирана частично на изложението в~\cite{Borovkov} и~\cite{DimitrovYanev}. За пълнота съм включил доказателства на основните свойства на пораждащи моментите и характеристична функции.

\subsection{Основни дефиниции и теореми}

\begin{definition}
  \uline{(Реална) случайна величина} над вероятностното пространство $(\Omega, \F, \Prob)$ наричаме всяка измерима функция $\xi : \Omega \to \R$.

  Условието за измеримост на $\xi$ може да се запише така: за всяко борелово множество $B \in \BorelAlgebra(\R)$ имаме
  \begin{displaymath}
    \xi^{-1} (B) = \{ \omega \in \Omega \mid \xi(\omega) \in B \} \in \F.
  \end{displaymath}

  \uline{Разпределение на $\xi$} наричаме мярката
  \begin{displaymath}
    \Prob_\xi(A) \coloneqq \Prob(\xi \in A).
  \end{displaymath}

  Две случайни величини $\xi$ и $\eta$ наричаме \uline{независими}, ако за всички $A, B \in \F$ е изпълнено
  \begin{displaymath}
    \Prob(\xi \in A, \eta \in B) = \Prob_\xi(A) \Prob_\eta(B).
  \end{displaymath}

  \uline{Функция на разпределение} на случайната величина $\xi$ наричаме
  \begin{displaymath}
    F_\xi(x) \coloneqq \Prob(\xi \leq x).
  \end{displaymath}

  Случайната величина $\xi$ наричаме \uline{абсолютно непрекъсната} и казваме, че $\xi$ има \uline{абсолютно непрекъснато разпределение}, ако функцията ѝ на разпределение е локално абсолютно непрекъсната в $\R$, т.е. абсолютно непрекъсната във всеки затворен интервал. Известно е, че абсолютно непрекъснатите в затворен интервал $[a, b]$ функции са точно тези, които са диференцируеми почти навсякъде в интервала, производните им в $[a, b]$ са интегруеми по Риман и за $x \in [a, b]$ е изпълнено
  \begin{displaymath}
    F_\xi(x) = \int_a^x F'_\xi(x) + F_\xi(a).
  \end{displaymath}

  Функцията $f_\xi: \R \mapsto \R$ наричаме \underline{вероятностна плътност} на случайната величина $\xi$, ако $F'_\xi(x) = f_\xi(x)$ във всяка точка, в която $F_\xi$ е диференцируема. Ако за едно разпределение съществуват множество плътности, те се различават само върху множество с лебегова мярка 0 и тъй като $f_\xi(x)$ се използва основно за интегриране, на практика няма значение с коя от плътностите ще работим.
\end{definition}

\begin{note}
  Абсолютно непрекъснатите случайни величини ще наричаме просто~\enquote{непрекъснати}. Понякога непрекъснати случайни величини се наричат такива с непрекъсната функция на разпределение, но това определение е прекалено общо и позволява т. нар. сингулярни разпределения, чиято плътност се анулира почти навсякъде.
\end{note}

\begin{proposition}[Основни свойства на функцията на разпределение]\label{thm:cdf-props}
  Функцията $F_\xi$ е функция на разпределение на някаква (не непременно абсолютно непрекъсната) случайна величина $\xi$ тогава и само тогава, когато са изпълнени
  \begin{enumerate}
    \item $F_\xi(x) \leq F_\xi(y), x < y$ (монотонност)
    \item $F_\xi(x)$ е непрекъсната отдясно
    \item $\lim_{x \downarrow -\infty} F_\xi(x) = 0$
    \item $\lim_{x \uparrow \infty} F_\xi(x) = 1$
  \end{enumerate}
\end{proposition}

Твърдение~\ref{thm:cdf-props} ни дава обосновка да задаваме случайни величини изцяло чрез функцията им на разпределение, т.е. без изрично да задаваме вероятностни пространства.

\begin{proof}[Доказателство на твърдение~\ref{thm:cdf-props}]
  ($\implies$)
  \begin{enumerate}
    \item За всички $x < y$
    \begin{multline*}
      F_\xi(x)
      =
      \Prob(\xi \leq x)
      =
      \Prob(\{ \omega \in \Omega \mid \xi(\omega) \leq x \})
      =
      \Prob(\{ \omega \in \Omega \mid \xi(\omega) \leq x \} \cap \{ \omega \in \Omega \mid \xi(\omega) \leq y \})
      \leq \\ \leq
      \Prob(\{ \omega \in \Omega \mid \xi(\omega) \leq y \})
      =
      \Prob(\xi \leq y)
      =
      F_\xi(y).
    \end{multline*}

    \item От монотонността на вероятностната мярка имаме
    \begin{multline*}
      \lim_{h \downarrow 0} F_\xi(x + h)
      =
      \lim_{h \downarrow 0} \Prob(\xi \leq x + h)
      =
      \lim_{h \downarrow 0} \Prob(\{ \omega \in \Omega \mid \xi(\omega) \leq x + h \})
      =
      \Prob(\cup_{h \geq 0} \{ \omega \in \Omega \mid \xi(\omega) \leq x + h \})
      = \\ =
      \Prob(\{ \omega \in \Omega \mid \xi(\omega) \leq x \})
      =
      \Prob(\xi \leq x)
      =
      F_\xi(x).
    \end{multline*}

    \item От монотонността на вероятностната мярка имаме
    \begin{multline*}
      \lim_{x \uparrow \infty} F_\xi(x)
      =
      \lim_{x \uparrow \infty} \Prob(\xi \leq x)
      =
      \lim_{x \uparrow \infty} \Prob(\{ \omega \in \Omega \mid \xi(\omega) \leq x \})
      =
      \Prob(\cup_{x \uparrow \infty} \{ \omega \in \Omega \mid \xi(\omega) \leq x \})
      = \\ =
      \Prob(\cup_{x \uparrow \infty} \{ \omega \in \Omega \mid \xi(\omega) \leq 1 \})
      =
      \Prob(\{ \omega \in \Omega \mid \xi(\omega) \leq \infty \})
      =
      \Prob(\Omega)
      =
      1.
    \end{multline*}

    \item $\lim_{x \uparrow \infty} F_\xi(x) = 0$ се доказва напълно аналогично на $\lim_{x \uparrow \infty} F_\xi(x) = 1$.
  \end{enumerate}

  ($\impliedby$) Нека функцията $F_\xi$ удовлетворява условията на теоремата. Дефинираме
  \begin{align*}
    \xi: \R \to \R,     &&& \Prob: \BorelAlgebra(\R) \to [0, 1], \\
    \xi(x) \coloneqq x, &&& \Prob((a, b]) \coloneqq F_\xi \left(\lim_{h \downarrow 0} b + h \right) - F_\xi(a), a < b \in \R.
  \end{align*}

  Интервалите от вида $(a, b]$ пораждат бореловата $\sigma$-алгебра $\BorelAlgebra(\R)$. Ще пропуснем доказателството на това, че $\Prob$ е вероятностна мярка над $(\R, \BorelAlgebra(\R))$.

  Тогава $\xi$ е измерима функция над $(\R, \BorelAlgebra(\R), \Prob)$ и освен това
  \begin{displaymath}
    \Prob(\xi \leq x)
    =
    \Prob((-\infty, x])
    =
    F_\xi(x)~\forall x \in \R.
  \end{displaymath}

  С други думи, построихме вероятностно пространство и случайна величина $\xi$, чиято функцията на разпределение е $F_\xi$.
\end{proof}

До края на темата ще считаме, че работим над вероятностното пространство $(\Omega, \F, \Prob)$.

\begin{theorem}\label{thm:density-props}
  Интегруемата по Риман функция $f_\xi: \R \to \R$ е плътност на някаква абсолютно непрекъсната случайна величина $\xi$ тогава и само тогава, когато са изпълнени
  \begin{enumerate}
    \item $f_\xi(x) \geq 0$ почти навсякъде (неотрицателност)
    \item $\int_\R f_\xi(x) dx = 1$ (нормираност)
  \end{enumerate}
\end{theorem}

Тази теорема ни позволява да задаваме непрекъсната случайна величина изцяло чрез плътността ѝ, поради което плътността понякога се нарича разпределение на случайната величина.

\begin{proof}[Доказателство на теорема~\ref{thm:density-props}]
  ($\implies$) Нека $f_\xi$ е плътност на $\xi$.
  \begin{enumerate}
    \item За всяка точка $x \in \R$, в която $F_\xi$ е диференцируема, имаме $f_\xi(x) = F'_\xi(x) = \lim_{h \downarrow 0} \frac {F_\xi(x + h) - F_\xi(x)} h \geq 0$ поради монотонността на $F_\xi$
    \item За произволно $c > 0$ функцията $F_\xi$ е абсолютно непрекъсната в $[-c, c]$. Следователно
    \begin{displaymath}
      \int_\R f_\xi(x) dx
      =
      \lim_{c \uparrow \infty} \int_{-c}^c f_\xi(x) dx
      =
      \lim_{c \uparrow \infty} F_\xi(c) - \lim_{c \to \infty} F_\xi(-c)
      =
      1 - 0.
    \end{displaymath}
  \end{enumerate}

  ($\impliedby$) Нека $f_\xi: \R \to \R$ е неотрицателна, интегруема по Риман и нормирана. Дефинираме функцията $F_\xi(x) \coloneqq \int_{-\infty}^x f_\xi(t) dt$. Ще покажем, че за $F_\xi$ са изпълнени свойствата на функция на разпределение:
  \begin{enumerate}
    \item Ако $x < y$, от адитивността на римановия интеграл и неотрицателността на $f_\xi$ следва
    \begin{displaymath}
      F_\xi(x)
      =
      \int_{-\infty}^x f_\xi(t) dt
      \leq
      \int_{-\infty}^x f_\xi(t) dt + \int_x^y f_\xi(t) dt
      =
      \int_{-\infty}^y f_\xi(t) dt
      =
      F_\xi(y).
    \end{displaymath}

    \item $F_\xi$ е непрекъсната отдясно, тъй като
    \begin{displaymath}
      \lim_{h \downarrow 0} F_\xi(x + h)
      =
      \lim_{h \downarrow 0} \int_{-\infty}^{x + h} f_\xi(t) dt
      =
      \int_{-\infty}^x f_\xi(t) dt + \lim_{h \downarrow 0} \int_x^{x + h} f_\xi(t) dt
      =
      \int_{-\infty}^x f_\xi(t) dt
      =
      F_\xi(x).
    \end{displaymath}

    \item От предположението за нормираност имаме
    \begin{displaymath}
      \lim_{x \uparrow \infty} F_\xi(x)
      =
      \lim_{x \uparrow \infty} \int_{-\infty}^x f_\xi(x) dx
      =
      \int_\R f_\xi(x) dx = 1.
    \end{displaymath}

    \item Директно пресмятаме
    \begin{displaymath}
      \lim_{x \downarrow -\infty} \int_{-\infty}^x f_\xi(x) dx
      =
      \int_{-\infty}^{-\infty} f_\xi(x) dx
      =
      0.
    \end{displaymath}
  \end{enumerate}

  Видяхме, че $F_\xi$ удовлетворява свойствата на функция на разпределение и по~\ref{thm:cdf-props} съществува случайна величина $\xi$, чиято плътност е $f_\xi$.

  Освен това $F_\xi$ е абсолютно непрекъсната във всеки затворен интервал $[a, b]$, тъй като тя има интегруема производна почти навсякъде и за $x \in [a, b]$ е изпълнено
  \begin{displaymath}
    F_\xi(x)
    =
    \int_{-\infty}^x f_\xi(x) dx
    =
    \int_{-\infty}^a f_\xi(x) dx + \int_a^x f_\xi(x) dx
    =
    F_\xi(a) + \int_a^x f_\xi(x) dx.
  \end{displaymath}
\end{proof}

\begin{proposition}[Конволюция на плътности]\label{thm:convolution}
  Сумата на две независими непрекъснати случайни величини $\xi$ и $\eta$ е непрекъсната случайна величина с плътност
  \begin{displaymath}
    f_{\xi + \eta} (x)
    =
    \int_\R f_\xi(t) f_\eta(x - t) dt
    =
    \int_\R f_\xi(t) f_\eta(x - t) dt.
  \end{displaymath}
\end{proposition}
\begin{proof} От формулата за пълната вероятност и независимостта на $\xi$ и $\eta$ имаме
  \begin{displaymath}
    F_{\xi + \eta} (x)
    =
    \Prob(\xi + \eta \leq x)
    =
    \Prob(\xi \leq x - \eta)
    =
    \int_\R \Prob(\xi \leq x - t) f_\eta(t) dt
    =
    \int_\R F_\xi(x - t) f_\eta(t) dt.
  \end{displaymath}

  Аналогично се доказва
  \begin{displaymath}
    f_{\xi + \eta} (x)
    =
    \int_\R f_\xi(t) f_\eta(x - t) dt.
  \end{displaymath}
\end{proof}

\begin{proposition}\label{thm:transformation-density}
  Ако $\xi$ е случайна величина и функцията $\psi: \R \to \R$ е строго монотонна и диференцируема, тогава $\psi(\xi)$ е непрекъсната случайна величина с плътност
  \begin{displaymath}
    f_{\psi(\xi)} (x)
    =
    \Abs{(\psi^{-1})'(x)} f(\psi^{-1}(x)).
  \end{displaymath}
\end{proposition}
\begin{proof} Диференцирайки $F_{\psi(\xi)} = \Prob(\psi(\xi) \leq x) = \Prob(\xi \leq \psi^{-1}(x)) = F_\xi(\psi^{-1}(x))$ по $x$ получаваме
  \begin{displaymath}
    f_{\psi(\xi)} (x)
    =
    F'_{\psi(\xi)} (x)
    =
    (F_{\xi} \circ \psi^{-1})'(x)
    =
    \Abs{(\psi^{-1})'(x)} f(\psi^{-1}(x)).
  \end{displaymath}
\end{proof}

\subsection{Очакване и моменти}

\begin{definition}
  Нека $\xi$ е непрекъсната случайна величина. Дефинираме \uline{очакване на $\xi$} чрез
  \begin{displaymath}
    \Expect(\xi) \coloneqq \int_\R x f_\xi(x) dx.
  \end{displaymath}
  Казваме, че $\xi$ има (крайно) очакване, ако интегралът е абсолютно сходящ, т.е. $\Abs{x f_\xi}$ е интегруема функция.

  Случайни величини с очакване нула наричаме \uline{центрирани}.

  Очакване от константа $x \in \R$ дефинираме да бъде самата константа $x$.
\end{definition}

\begin{note}
  Очакването се дефинира за произволна случайна величина $\xi$ се дефинира чрез интеграл по вероятностната мярка, т.е.
  \begin{displaymath}
    \Expect(\xi) \coloneqq \int \xi d\Prob.
  \end{displaymath}

  Нещо повече, очакването е линеен функционал над линейното пространството от всички случайни величини над $(\Omega, \F, \Prob)$.

  Непрекъснатите случайни величини обаче не образуват линейно подпространство, тъй като сумата на две зависими непрекъснати случайни величини може да не бъде непрекъсната (например $\xi - \xi = 0$). Тъй като тук се ограничаваме само до непрекъснати случайни величини, ще формулираме някои свойства (например адитивност) само в частния случай, в който случайните величини са независими.
\end{note}

\begin{proposition}\label{thm:expect-product}
  За независими непрекъснати случайни величини $\xi$ и $\eta$ с крайно очакване е изпълнено
  \begin{displaymath}
    \Expect(\xi \eta) = \Expect(\xi) \Expect(\eta).
  \end{displaymath}
\end{proposition}
\begin{proof}
  Прилагаме теоремата на Фубини, теоремата за средните стойности и формулата за пълната вероятност:
  \begin{multline*}
    \Expect(\xi \eta)
    =
    \int_\R z f_{\xi \eta}(z) dz
    =
    \int_\R z \Prob(z \leq \xi \eta < z + dz) dz
    =
    \int_\R \int_\R z \Prob(z \leq t \xi < z + dz) f_\eta(t) dz
    = \\ =
    \int_\R \int_\R z \left( F_\xi\left(\frac z t \right) - F_\xi\left(\frac z t + d\frac z t \right) \right) f_\eta(t) dt \; dz
    =
    \int_\R \int_\R z f_\xi \left(\frac z t \right) f_\eta(t) dt \; dz
    = \\ =
    \int_\R t f_\eta(t) \left(\int_\R \frac z t f_\xi \left(\frac z t \right) d\frac z t \right) dt
    =
    \Expect(\xi) \int_\R t f_\eta(t) dt
    =
    \Expect(\xi) \Expect(\eta).
  \end{multline*}
\end{proof}

\begin{proposition}\label{thm:expect-additive}
  Ако $\xi$ и $\eta$ са непрекъснати и независими, имаме $\Expect(\xi + \eta) = \Expect(\xi) + \Expect(\eta)$.
\end{proposition}
\begin{proof}
  От твърдение~\ref{thm:convolution} знаем, че $\xi + \eta$ също има непрекъснато разпределение и плътността ѝ е конволюция на плътностите на $\xi$ и $\eta$. Тогава от теоремата на Фубини имаме
  \begin{multline*}
    \Expect(\xi + \eta)
    =
    \int_\R x f_{\xi + \eta} (x) dx
    =
    \int_\R x \int_\R f_\xi(x - t) f_\eta(t) dt \; dx
    =
    \int_\R \left( \int_\R x f_\xi(x - t) dx \right) f_\eta(t) dt
    = \\ =
    \int_\R \left( \int_\R (x - t + t) f_\xi(x - t) d(x - t) \right) f_\eta(t) dt
    = \\ =
    \int_\R \left( \int_\R (x - t) f_\xi(x - t) d(x - t) + t \int_\R f_\xi(x - t) d(x - t) \right) f_\eta(t) dt
    = \\ =
    \int_\R (\Expect(\xi) + t) f_\eta(t) dt
    =
    \Expect(\xi) \int_\R f_\eta(t) dt + \int_\R t f_\eta(t) dt
    =
    \Expect(\xi) + \Expect(\eta).
  \end{multline*}
\end{proof}

\begin{proposition}\label{thm:lotus}
  Нека $\xi$ е непрекъсната случайна величина с крайно очакване. Нека $\psi: \R \to \R$ е монотонна. Тогава $\psi(\xi)$ е непрекъсната случайна и е изпълнено
  \begin{displaymath}
    \Expect(\psi(\xi))
    =
    \int_\R \psi(x) f_\xi(x) dx.
  \end{displaymath}
\end{proposition}

\begin{proof}
  \begin{multline*}
    \Expect(\psi(\xi))
    =
    \int_\R x f_{\psi(\xi)}(x) dx
    =
    \int_\R x \Prob(x \leq \psi(\xi) \leq x + dx) dx
    = \\ =
    \int_\R x \Prob(\psi^{-1}(x) \leq \xi \leq \psi^{-1}(x + dx)) dx
    =
    \int_\R x \left( F_\xi(\psi^{-1}(x + dx)) - F_\xi(\psi^{-1}(x)) \right) dx
    = \\ =
    \int_\R x f_\xi(\psi^{-1}(x)) dx
    =
    \int_\R \psi(x) f_\xi(x) d \psi(x)
    =
    \int_\R \psi(x) f_\xi(x) \psi'(x) dx.
  \end{multline*}
\end{proof}

Доказаните в твърдения~\ref{thm:expect-product},~\ref{thm:expect-additive} и~\ref{thm:lotus} свойства на очакването значително опростяват работата с него.

\begin{definition}
  \uline{Ковариация на случайните величини $\xi$ и $\eta$} наричаме
  \begin{displaymath}
    \Cov(\xi, \eta)
    \coloneqq
    \Expect((\xi - \Expect \xi) (\eta - \Expect \eta)).
  \end{displaymath}

  \uline{Дисперсия или вариация на случайната величина $\xi$} наричаме
  \begin{displaymath}
    \Var(\xi)
    \coloneqq
    \Cov(\xi, \xi)
    =
    \Expect \left({(\xi - \Expect \xi)}^2 \right)
    =
    \Expect(\xi^2 - 2 \xi \Expect \xi + {\Expect(\xi)}^2)
    =
    \Expect(\xi^2) - 2 {\Expect(\xi)}^2 + {\Expect(\xi)}^2
    =
    \Expect(\xi^2) - {\Expect(\xi)}^2.
  \end{displaymath}

  Числото $\Expect(\xi^n)$ наричаме \uline{$n$-ти момент на $\xi$}, а $\Expect \left( {(\xi - \Expect \xi)}^n \right)$ наричаме \uline{$n$-ти централен момент на $\xi$}.

  Очакването всъщност е просто първият момент, а дисперсията - вторият централен момент. Коренът на дисперсията се нарича \uline{стандартно отклонение} и често се бележи със $\sigma_\xi$.

  Две случайни величини се наричат \uline{ортогонални}, ако ковариацията им е $0$, защото ковариацията играе ролята на скаларно произведение в пространството $\LSpace^2$ от (всички, не непременно непрекъснати) случайни величини с краен втори момент.

  Случайни величини със стандартно отклонение единица наричаме \uline{нормирани}, тъй като стандартно отклонение играе ролята на норма в $\LSpace^2$.
\end{definition}

\begin{proposition}\label{thm:orthogonal-if-independent}
  Ако две случайни величини са независими и имат крайно очакване, те са ортогонални.
\end{proposition}
\begin{proof}
  Нека $\xi$ и $\eta$ са независими и имат крайни очаквания съответно $\mu$ и $\eta$. Тогава
  \begin{displaymath}
    \Cov(\xi, \eta)
    =
    \Expect((\xi - \mu) (\eta - \nu))
    =
    \Expect(\xi - \mu) \Expect(\eta - \nu)
    =
    (\mu - \mu) (\nu - \nu)
    =
    0 \cdot 0
    =
    0.
  \end{displaymath}
\end{proof}

\begin{proposition}\label{thm:lower-order-moments}
  Ако $\Expect(\xi^n)$ съществува, съществуват и моментите от по-нисък ред.
\end{proposition}
\begin{proof}
  Първо да забележим, че за $y \in (0, 1)$ имаме ${\Prob(\xi \leq x)}^y < \Prob(\xi \leq x)$. Ще докажем, че $\Expect({\Abs{\xi}}^{n-1})$ съществува. Прилагаме неравенството на Йенсен за риманови интеграли:
  \begin{multline*}
    \Expect({\Abs{\xi}}^{n-1})
    \leq
    {\Expect({\Abs{\xi}}^{n-1})}^{\frac n {n-1}}
    =
    {\left( \int_\R {\Abs{x_k}}^{n-1} f_\xi(x) \right)}^{\frac n {n-1}}
    \leq
    \int_\R {\left({\Abs{x_k}}^{n-1} f_\xi(x) \right)}^{\frac n {n-1}}
    < \\ <
    \int_\R {\left({\Abs{x_k}}^{n-1}\right)}^{\frac n {n-1}} f_\xi(x)
    =
    \int_\R {\Abs{x_k}}^n f_\xi(x)
    =
    \Expect({\Abs{\xi}}^n).
  \end{multline*}
\end{proof}

\printbibliography

\end{document}
