\documentclass[numbers=endperiod, bibliography=totocnumbered]{scrartcl}

\usepackage{../../common/base_math_packages}
\usepackage{../../common/fonts}
\usepackage{../../common/base_packages}
\usepackage{../../common/macros}
\usepackage{../../common/theorem_styles}

% Custom packages
\usepackage{../../common/macros}
\usepackage{../../common/theorem_styles}

% Misc packages
\usepackage{enumitem} % Customization of enum counters

% Bibliography
\addbibresource{./references.bib}

% Document
\title{Тема 18}
\subtitle{Поасонов процес. Характеризационни свойства. Приложения.}
\author{Янис Василев, \Email{ianis@ivasilev.net}}
\date{26 юни 2019}

\begin{document}

\maketitle

\section{Теория}

Теорията е базирана на~\cite{Lectures}.

\subsection{Анотация}

Изложената анотацията е взета от конспекта~\cite{Syllabus} за 2018г.

\begin{enumerate}
  \item Дефиниция за броящ процес
  \item Дефиниция за Поасонов процес
  \item Връзка с експоненциално разпределение
  \item Характеризационни свойства - разпределение на времето за чакане и условни разпределения на времето и моментите на появяване
  \item Пример с интерпретация на горепосочените свойства
  \item Нехомогенен и сложен Поасонов процес
\end{enumerate}

\subsection{Основни понятия}

Ще считаме, че е фиксирано някакво вероятностно пространство \( (\Omega, \F, \Prob) \) и множество \( \mathcal{N} \) от целочислени случайни величини с неотрицателни стойности над това пространство.

\begin{definition}
  \underLine{Броящ процес} наричаме монотонната с вероятност \( 1 \) функция \( N: \RNNeg \to \mathcal{N} \), чиито аргументи обикновено интерпретираме като време и бележим с \( t \), а стойностите \( N(t) \) като брой настъпили събития за време \( t \).

  \underLine{Нарастване} на процеса с \( h > 0 \) в момента \( t \) наричаме разликите \( N(t+h) - N(t) \). Ако всички нараствания за един процес са независими, казваме, че процесът е с \underLine{независими нараствания}. Ако разпределенията на нарастванията \( N(t_1+h) - N(t_1) \) и \( N(t_2+h) - N(t_2) \) зависят само от \( h>0 \), казваме, че процесът \( N(t) \) е със \underLine{стационарни нараствания}.
\end{definition}

\begin{definition}
  Ще дадем три еквивалентни определения за \underLine{Поасонов процес}.

  \begin{enumerate}
    \item\label{def:pp-1} Броящият процес \( N(t), t \geq 0 \) наричаме \underLine{Поасонов със степен \( \lambda > 0 \)}, ако са изпълнени
    \begin{enumerate}
      \item \( N(0) = 0 \)
      \item \( N(t) \) е процес с независими нараствания
      \item Нарастванията \( N(t+h)-N(t) \) имат Поасоново разпределение със степен \( h\lambda \), т.е.
      \begin{align*}
        \Prob(N(t+h)-N(t) = k)
        =
        \frac{e^{-h\lambda} {(h\lambda)}^k} {k!}.
      \end{align*}
    \end{enumerate}

    \item\label{def:pp-2} Броящият процес \( N(t), t \geq 0 \) наричаме \underLine{Поасонов със степен \( \lambda > 0 \)}, ако са изпълнени
    \begin{enumerate}
      \item \( N(0) = 0 \)
      \item \( N(t) \) е процес със стационарни и независими нараствания
      \item \( \frac {\Prob(N(h) = 1)} h \Conv[h \to 0]{} \lambda \)
      \item \( \frac {\Prob(N(h) \geq 1)} h \Conv[h \to 0]{} 0 \)
    \end{enumerate}

    \item\label{def:pp-3} Нека \( \xi_1, \xi_2, \ldots \) са независими случайни величини с разпределение \( \DExp(\lambda) \), т.е.
    \begin{align*}
      f_{\xi_n}(t) =
      \begin{cases}
        \lambda e^{-\lambda t}, &t \geq 0 \\
        0, &t < 0
      \end{cases},
      n = 1, 2, \ldots.
    \end{align*}

    Полагаме
    \begin{align*}
      S_0 &\coloneqq 0, \\
      S_n &\coloneqq \sum_{k=1}^n \xi_k.
    \end{align*}

    Очевидно \( S_n \in \DGamma(n, \lambda), n > 0 \), т.е.
    \begin{align*}
      f_{S_n}(t)
      =
      \begin{dcases}
        \frac {\lambda^n t^{n-1} e^{-\lambda t}} {\Gamma(n)} = \lambda e^{-\lambda t} \frac {{(\lambda t)}^{n-1}} {(n-1)!}, &t \geq 0, \\
        0, &t < 0,
      \end{dcases}
      n = 1, 2, \ldots.
    \end{align*}

    Броящият процес \( N(t) \), за който е изпълнено
    \begin{align*}
      N(t) \coloneqq \max \{ m \geq 0 \mid S_m \leq t \},
    \end{align*}
    се нарича \underLine{Поасонов процес със степен \( \lambda \)}, случайните величини \( S_0, S_1, S_2, \ldots \) наричаме времена на появява на \( n \)-тото събитие, а \( \xi_1, \xi_2, \ldots \) наричаме \underLine{времена на чакане за поява на \( n \)-тото събитие}.
  \end{enumerate}
\end{definition}

\begin{note}
  \mbox{}
  \begin{enumerate}
    \item И при двете определения поасоновият процес \( N(t) \) има стационарни нараствания.
    \item Параметърът \( \lambda \) се нарича степен на процеса, тъй като \( \Expect(N(t)) = \lambda t \).
  \end{enumerate}
\end{note}

\begin{theorem}
  Трите определения са еквивалентни.
\end{theorem}
\begin{proof}
  (\ref{def:pp-1} \( \implies \)~\ref{def:pp-2}) Нека първо броящият процес \( N(t) \) удовлетворява първото определение.

  Тогава
  \begin{align*}
    \frac {\Prob(N(h) = 1)} h
    =
    \frac 1 h \frac {e^{-\lambda h} {(\lambda h)}^1} {1!}
    =
    \lambda e^{-\lambda h}
    \Conv[h \to 0]{}
    \lambda.
  \end{align*}
  и
  \begin{align*}
    \frac {\Prob(N(h) \geq 1)} h
    &=
    \frac {1 - \Prob(N(h) < 2)} h
    = \\ &=
    \frac {1 - \Prob(N(h) = 1) - \Prob(N(h) = 0)} h
    = \\ &=
    \frac {1 - \lambda h e^{-\lambda h} - e^{-\lambda h}} h
    = \\ &=
    \frac {1 - e^{-\lambda h}} h - \lambda e^{-\lambda h}
    = \\ &=
    - \frac {e^{-\lambda (0 + h)} - e^{-\lambda 0}} h - \lambda e^{-\lambda h}
    \Conv[h \to 0]{}
    -(-\lambda) e^{-\lambda 0} - \lambda e^{-\lambda 0}
    =
    0.
  \end{align*}

  (\ref{def:pp-2} \( \implies \)~\ref{def:pp-1}) Обратно, ако \( N(t) \) удовлетворява второто определение, получаваме диференциалното уравнение
  \begin{align*}
    \frac {\partial \Prob(N(t) = 0)} {\partial t}
    =&
    \lim_{h \to 0} \frac {\Prob(N(t+h) = 0) - \Prob(N(t) = 0)} h
    = \\ =&
    \lim_{h \to 0} \frac {\Prob(N(t+h) - N(t) = 0, N(t) - N(0) = 0) - \Prob(N(t) = 0)} h
    = \\ =&
    \lim_{h \to 0} \frac {\Prob(N(h) = 0, N(t) = 0) - \Prob(N(t) = 0)} h
    = \\ =&
    \Prob(N(t) = 0) \lim_{h \to 0} \frac {\Prob(N(h) = 0) - 1} h
    = \\ =&
    \Prob(N(t) = 0) \lim_{h \to 0} \frac {1 - \Prob(N(h) = 1) - \Prob(N(h) \geq 1) - 1} h
    = \\ =&
    - \Prob(N(t) = 0) \lim_{h \to 0} \frac {\Prob(N(h) = 1) + \Prob(N(h) \geq 1)} h
    = \\ =&
    - \Prob(N(t) = 0) \cdot (\lambda + 0)
    = \\ =&
    -\lambda \Prob(N(t) = 0).
  \end{align*}

  Като се има предвид началното условие \( \Prob(N(0) = 0) = 1 \), това уравнение има решение
  \begin{align*}
   \Prob(N(t) = 0) = e^{-\lambda t}.
  \end{align*}

  За \( n > 0 \) формулата за пълната вероятност ни дава
  \begin{align*}
    \Prob(N(t+h) = n)
    =&
    \sum_{k=0}^n \Prob(N(t+h) = n, N(t) = k)
    = \\ =&
    \sum_{k=0}^n \Prob(N(t+h) - N(t) = n - k, N(t) = k)
    = \\ =&
    \sum_{k=0}^n \Prob(N(h) = n - k, N(t) = k)
    = \\ =&
    \sum_{k=0}^n \Prob(N(h) = n - k) \Prob(N(t) = k)
    = \\ =&
    \sum_{k=0}^n \Prob(N(h) = k) \Prob(N(t) = n - k),
  \end{align*}
  откъдето намираме производната по \( t \)
  \begin{align*}
    &\frac {\partial \Prob(N(t) = n)} {\partial t}
    = \\ =&
    \lim_{h \to 0} \frac {\Prob(N(t+h) = n) - \Prob(N(t) = n)} h
    = \\ =&
    \lim_{h \to 0} \frac 1 h \left( \sum_{k=0}^n \Prob(N(h) = k) \Prob(N(t) = n - k) - \Prob(N(t) = n) \right)
    = \\ =&
    \lim_{h \to 0} \frac 1 h [\Prob(N(t) = n) [\Prob(N(h) = 0) - 1] + \Prob(N(h) = 1) \Prob(N(t) = n - 1) + \cdots]
    = \\ =&
    \lim_{h \to 0} \frac {\Prob(N(t) = n) [\Prob(N(h) = 0) - 1]} h + \lambda \Prob(N(t) = n - 1) + 0
    = \\ \overset {\text{L'Hospital}} =&
    \lim_{h \to 0} \frac {\Prob(N(t) = n) \cdot [-\lambda \Prob(N(h) = 0)]} 1 + \lambda \Prob(N(t) = n - 1) + 0
    = \\ =&
    -\lambda \Prob(N(t) = n) + \lambda \Prob(N(t) = n - 1).
  \end{align*}

  Тогава
  \begin{align*}
    \frac {\partial \Prob(N(t) = n)} {\partial t}
    =&
    -\lambda \Prob(N(t) = n) + \lambda \Prob(N(t) = n - 1),
    \\
    e^{\lambda t} \frac {\partial \Prob(N(t) = n)} {\partial t}
    =&
    -\lambda e^{\lambda t} \Prob(N(t) = n) + \lambda e^{\lambda t} \Prob(N(t) = n - 1)
    \\
    e^{\lambda t} \left(\frac {\partial \Prob(N(t) = n)} {\partial t} + \lambda \Prob(N(t) = n) \right)
    =&
    \lambda e^{\lambda t} \Prob(N(t) = n - 1),
    \\
    \frac {\partial [e^{\lambda t} \Prob(N(t) = n)]} {\partial t}
    =&
    \lambda e^{\lambda t} \Prob(N(t) = n - 1).
  \end{align*}

  За \( n = 1 \) интегрираме уравнението и получаваме
  \begin{align*}
    \frac {\partial [e^{\lambda t} \Prob(N(t) = 1)]} {\partial t}
    =&
    \lambda e^{\lambda t} \Prob(N(t) = 0)
    =
    \lambda e^{\lambda t} e^{-\lambda t}
    =
    \lambda,
    \\
    e^{\lambda t} \Prob(N(t) = 1)
    =&
    \lambda t + C,
    \\
    \Prob(N(t) = 1)
    =&
    \lambda t e^{-\lambda t} + C.
  \end{align*}

  Понеже \( \Prob(N(0) = 1) = 0 \), имаме \( C = 0 \) и
  \begin{align*}
    \boxed{\Prob(N(t) = 1) = \lambda t e^{-\lambda t}}.
  \end{align*}

  С индукция по \( n \) получаваме функцията на вероятностите на познатото Поасоново разпределение със степен \( \lambda t \)
  \begin{align*}
    \frac {\partial [e^{\lambda t} \Prob(N(t) = n)]} {\partial t}
    =&
    \lambda e^{\lambda t} \Prob(N(t) = n - 1)
    =
    \lambda e^{\lambda t} e^{-\lambda t} \frac {{(\lambda t)}^{n-1}} {(n-1)!}
    =
    \frac {\lambda^n} {(n-1)!} t^{n-1},
    \\
    e^{\lambda t} \Prob(N(t) = n)
    =&
    \frac {{(\lambda t)}^n} {n!} + C,
    \\
    \Prob(N(t) = n)
    =&
    e^{-\lambda t} \frac {{(\lambda t)}^n} {n!} + C.
  \end{align*}

  Понеже \( \Prob(N(0) = n) = \delta_{n,0} \), имаме \( C = 0 \) и
  \begin{align*}
    \boxed{\Prob(N(t) = n) = e^{-\lambda t} \frac {{(\lambda t)}^n} {n!}}.
  \end{align*}

  (\ref{def:pp-1} \( \implies \)~\ref{def:pp-3}) Отново приемаме първото определение и разглеждаме вероятността за нарастване между моментите \( t \) и \( t+h \), т.е.
  \begin{align*}
    \Prob(N(t) < N(t+h))
    &=
    \Prob(0 < N(t+h) - N(t))
    = \\ &=
    \Prob(0 < N(h))
    = \\ &=
    1 - \Prob(N(h) = 0)
    = \\ &=
    1 - e^{-\lambda h}.
  \end{align*}

  Виждаме, времето за чакане за настъпване на събитие между моментите \( t \) и \( t+h \) е експоненциално разпределено и не зависи от \( t \), т.е. времената за чакане са независими.

  (\ref{def:pp-3} \( \implies \)~\ref{def:pp-1}) Сега приемаме третото определение и разглеждаме вероятността да са се случили \( n \) събития до момента \( t \)
  \begin{align*}
    \Prob(N(t) = n)
    &=
    \Prob(\max \{ m \geq 0 \mid S_m \leq t \} = n)
    = \\ &=
    \Prob(S_n \leq t < S_{n+1})
    = \\ &=
    \Prob(S_n \leq t < S_n + \xi_{n+1})
    = \\ &=
    \int_0^t \Prob(t < x + \xi_{n+1}) f_{S_n} (x) dx
    = \\ &=
    \int_0^t (1 - F_{\xi_{n+1}} (t - x)) \cdot f_{S_n} (x) dx
    = \\ &=
    \int_0^t e^{-\lambda (t-x)} \cdot \lambda e^{-\lambda x} \frac {{(\lambda x)}^{n-1}} {(n-1)!} dx
    = \\ &=
    e^{-\lambda t} \frac 1 {(n-1)!} \int_0^t {(\lambda x)}^{n-1} d(\lambda x)
    = \\ &=
    \boxed{e^{-\lambda t} \frac {{(\lambda t)}^n} {n!}},
  \end{align*}
  което е точно функция на вероятностите на Поасоново разпределение със степен \( \lambda t \).
\end{proof}

От експоненциалното разпределение се наследява следното свойство:
\begin{proposition}[Липса на памет]
  За Поасонов процес \( N(t) \) със степен \( \lambda > 0 \) е изпълнено
  \begin{align*}
    \Prob(N(t+h) > n + m \mid N(t) > n)
    =
    \Prob(N(h) > m)
  \end{align*}
\end{proposition}
\begin{proof}
  \begin{align*}
    \Prob(N(t+h) = n + m \mid N(t) = n)
    &=
    \frac {\Prob(N(t+h) = n + m, N(t) = n)} {\Prob(N(t) = n)}
    = \\ &=
    \frac {\Prob(N(t+h) - N(t) = m, N(t) = n)} {\Prob(N(t) = n)}
    = \\ &=
    \frac {\Prob(N(h) = m, N(t) = n)} {\Prob(N(t) = n)}
    = \\ &=
    \frac {\Prob(N(h) = m) \Prob(N(t) = n)} {\Prob(N(t) = n)}
    = \\ &=
    \Prob(N(h) = m).
  \end{align*}
\end{proof}

\subsection{Условни разпределения за времето на появяване}

Считаме, че е фиксиран някакъв Поасонов процес \( N(t) \) със степен \( \lambda > 0 \).

\begin{definition}
  Ако \( \eta_1, \ldots, \eta_n \) са наблюдения над случайна величина \( \eta \), \underLine{наредени статистики} наричаме елементите на вариационния ред \( \eta_{(1)} \leq \cdots \leq \eta_{(n)} \).
\end{definition}

\begin{proposition}
  Ако \( \eta_1, \ldots, \eta_n \) са независими наблюдения над непрекъсната случайна величина \( \eta \), съвместната плътност на наредените им статистики има вида
  \begin{align*}
    f_{\eta_{(1)}, \ldots, \eta_{(n)}} (x_1, \ldots, x_n)
    =
    \begin{cases}
      n! f_{\eta_{1}, \ldots, \eta_{n}} (x_1, \ldots, x_n), &x_1 < \cdots < x_n, \\
      0, &\text{ иначе}
    \end{cases}
  \end{align*}
\end{proposition}
\begin{proof}
  Ако числата \( x_1, \ldots, x_n \) не са наредени, не е възможно наредените статистики да приемат тази комбинация от стойности и затова съвместната им плътност се анулира.

  Ако те са наредени, съвместната плътност на съответните наредени статистики ще се различава от съвместната плътност на наблюденията с нормираща константа. Тъй като има \( n \)! начина да наредим \( n \) различни числа, тази нормираща константа е \( n \)!.
\end{proof}

\begin{note}
  В частния случай, когато \( \eta \in \DUniform(0, t) \), съвместната плътност на наредените статистики ще бележим с \( u_t(x_1, \ldots, x_n) \). Тази плътност има вида
  \begin{align*}
    u_t(x_1, \ldots, x_n)
    =
    \begin{cases}
      n! t^{-n}, &x_1 < \cdots < x_n, \\
      0, &\text{ иначе}
    \end{cases}
  \end{align*}
\end{note}

\begin{theorem} Условната съвместна плътност \( f_{S_1, \ldots, S_n}(x_1, \ldots, x_n \mid N(t) = n) \) на \( n \) последователни момента на появяване до момента \( t \) съвпада с \( u_t(x_1, \ldots, x_n) \).
\end{theorem}
\begin{proof}
  Първо да забележим, че
  \begin{align*}
    f_{S_1, \ldots, S_n}(x_1, \ldots, x_n \mid N(t) = n)
    =
    \begin{dcases}
      \frac {f_{S_1, \ldots, S_n, S_{n+1}}(x_1, \ldots, x_n, t)} {\Prob(N(t) = n)}, &x_n < t \\
      0, &\text{ иначе}
    \end{dcases}
  \end{align*}

  Тъй като \( S_{k-1} = S_k - \xi_k, k > 1 \), за \( x_1 < \cdots < x_n < t \) имаме
  \begin{align*}
    f_{S_1, \ldots, S_n}(x_1, \ldots, x_n \mid N(t) = n)
    &=
    \frac {f_{\xi_1, \xi_2, \ldots, \xi_n}(x_1, x_2 - x_1, \ldots, x_n - x_{n-1}, t - x_n)} {\Prob(N(t) = n)}
    = \\ &=
    \frac {f_{\xi_1}(x_1) f_{\xi_1}(x_2 - x_1) \ldots f_{\xi_{n-1}}(x_n - x_{n-1}) f_{\xi_n}(t - x_n)} {\Prob(N(t) = n)}
    = \\ &=
    \frac {\lambda^n e^{-\lambda x_1} e^{-\lambda(x_2 - x_1)} \ldots e^{-\lambda(x_n - x_{n-1})} e^{-\lambda(t - x_n)}} {e^{-\lambda t} \frac {{(\lambda t)}^n} {n!}}
    = \\ &=
    \frac {n!} {t^n} e^{- \lambda (-t + t - x_n + x_n \ldots x_2 - x_1 + x_1)}
    =
    \frac {n!} {t^n}
    =
    u_t(x_1, \ldots, x_n).
  \end{align*}
\end{proof}

\subsection{Приложения}

\subsubsection{Моделиране на надеждност}

Машинен елемент се поврежда средно три пъти годишно и моментално се заменя с нов. Нека случайните величини \( \xi_1, \xi_2, \ldots \) описват времената на живот на последователно използвани елементи.

Предполагаме, че те са експоненциално разпределени с параметър \( 3 \). Тогава времето на повреда на \( n \)-тия машинен елемент се описва от случайните величини \( S_n = \xi_1 + \cdots + \xi_n \), а
\begin{align*}
  N(t) \coloneqq \max \{ m \geq 0 \mid S_m \leq t \}
\end{align*}
е Поасонов процес.

Този модел ни позволява да отговаряме на много въпроси, сред които:
\begin{enumerate}
  \item Количеството повредени машинни елемента за три години имат разпределение \( N(3) \in \DPoisson(9) \). За три години се очакват средно \( 9 \) повреди със стандартно отклонение \( 3 \), а вероятността да се повредят цели \( 12 \) елемента е
  \begin{align*}
    \Prob(N(3) = 12) = e^{-9} \frac {9^{12}} {12!} \approx 0.07.
  \end{align*}

  \item Наредените времена за повреда на последните \( n \) машинни елементи са равновероятни.
  \item Поради липсата на памет на процеса, вероятността за нови повреди не се влияе от зачестяване на броя повреди за единица време.
\end{enumerate}

\subsubsection{Моделиране на интернет трафик}

Можем да използваме Поасонов процес, за да моделираме броя заявки към даден сървър. Ако за една секунда се случват средно \( 1000 \) заявки, използваме Поасонов процес с параметър \( \lambda = 1000 \).
\begin{enumerate}
  \item Редно е да очакваме \( \Expect(N(90)) \pm 2\sqrt{\Var(N(90))} = 90000 \pm 600 \) заявки за минута и половина.
  \item Наредените времена на заявките са равновероятни.
  \item Поради липсата на памет на процеса, вероятността за нови повреди не се влияе от броя заявки до момента.
\end{enumerate}

\subsection{Обобщения}

\subsubsection{Сложен Поасонов процес}

\begin{definition}
  Ако \( N(t) \) е Поасонов процес и \( \eta_1, \eta_2, \ldots \) са независими помежду си и от \( N(t) \) неотрицателни целочислени случайни величини с еднакво разпределение с, процесът
  \begin{align*}
    X(t) = \sum_{k=1}^{N(t)} \eta_k, t \geq 0,
  \end{align*}
  се нарича \underLine{сложен Поасонов}.
\end{definition}

\begin{note}
  Ако \( \eta_k \) са изродени в \( 1 \) случайни величини, получаваме обикновения Поасонов процес. Сложният Поасонов процес ни позволява лесно да моделираме настъпването на цели групи от събития, например пристигане на групи от хора с влак.
\end{note}

\begin{theorem}[Тъждество на Валд]
  За очакването и дисперсията на сложен Поасонов процес имаме
  \begin{align*}
    \Expect(X(t)) = \Expect(N(t)) \Expect(\xi_1),
    &&
    \Var(X(t)) = \Expect(N(t)) \Expect(\xi_1^2).
  \end{align*}
\end{theorem}
\begin{proof}
  \begin{align*}
    \Expect(X(t))
    &=
    \Expect\left( \sum_{k=1}^{N(t)} \eta_k \right)
    = \\ &=
    \sum_{n=0}^\infty \Expect\left( \sum_{k=1}^{N(t)} \eta_k \mid N(t) = n \right) \Prob(N(t) = n)
    = \\ &=
    \sum_{n=0}^\infty \Prob(N(t) = n) \sum_{k=1}^n \Expect(\eta_k)
    = \\ &=
    \Expect(\eta_1) \sum_{n=0}^\infty n \Prob(N(t) = n)
    = \\ &=
    \boxed{\Expect(\eta_1) \Expect(N(t))},
    \\
    \Expect({X(t)}^2)
    &=
    \Expect\left( {\left(\sum_{k=1}^{N(t)} \eta_k \right)}^2 \right)
    = \\ &=
    \Expect\left( \sum_{k=1}^{N(t)} \sum_{m=1}^{N(t)} \eta_k \eta_m \right)
    = \\ &=
    \sum_{n=0}^\infty \Expect\left( \sum_{k=1}^{N(t)} \sum_{m=1}^{N(t)} \eta_k \eta_m \mid N(t) = n \right) \Prob(N(t) = n)
    = \\ &=
    \sum_{n=0}^\infty \left( \sum_{k=1}^n \sum_{m=1}^n \Expect(\eta_k \eta_m) \right) \Prob(N(t) = n)
    = \\ &=
    \sum_{n=0}^\infty \left( n \Expect(\eta_1^2) + n (n-1) {\Expect(\eta_1)}^2 \right) \Prob(N(t) = n)
    = \\ &=
    \sum_{n=0}^\infty n \left( \Expect(\eta_1^2) - {\Expect(\eta_1)}^2 \right) \Prob(N(t) = n) + \sum_{n=0}^\infty n^2 \left( {\Expect(\eta_1)}^2 \right) \Prob(N(t) = n)
    = \\ &=
    \boxed{\Var(\eta_1) \Expect(N(t)) + {\Expect(\eta_1)}^2 \Expect({N(t)}^2)},
    \\
    \Var(X(t))
    &=
    \Var(\eta_1) \Expect(N(t)) + {\Expect(\eta_1)}^2 \Expect({N(t)}^2) - {\Expect(\eta_1)}^2 {\Expect(N(t))}^2
    = \\ &=
    \Var(\eta_1) \Expect(N(t)) + {\Expect(\eta_1)}^2 \Var(N(t))
    = \\ &=
    (\Var(\eta_1) + {\Expect(\eta_1)}^2) \lambda t
    = \\ &=
    \lambda t \Expect(\eta_1^2)
    = \\ &=
    \boxed{\Expect(N(t)) \Expect(\eta_1^2)}.
  \end{align*}
\end{proof}

\subsubsection{Нехомогенен Поасонов процес}

Ако допуснем степента \( \lambda(t) \) да варира с времето нарастванията няма да бъдат стационарни и ще получим следното обобщение на определение~\ref{def:pp-2} за обикновения Поасонов процес:
\begin{definition}
  Броящият процес \( N(t), t \geq 0 \) наричаме \underLine{нехомогенен Поасонов със степен \( \lambda(t) \geq 0 \)}, ако са изпълнени
  \begin{enumerate}
    \item \( N(0) = 0 \)
    \item \( N(t) \) е процес с независими нараствания
    \item \( \frac {\Prob(N(t+h) - N(t) = 1)} h \Conv[h \to 0]{} \lambda(t) \)
    \item \( \frac {\Prob(N(t+h) - N(t) \geq 1)} h \Conv[h \to 0]{} 0 \)
  \end{enumerate}

  За удобство полагаме
  \begin{align*}
    \Lambda(t) \coloneqq \int_0^t \lambda(x) dx.
  \end{align*}
\end{definition}

Формулираме следното обобщение на свойствата на обикновения Поасонов процес, чиито доказателства са аналогични:
\begin{theorem}
  Нека \( N(t) \) е нехомогенен Поасонов процес със степен \( \lambda(t) \). Тогава
  \begin{enumerate}
    \item \( N(t) \in \DPoisson(\Lambda(t)) \).
    \item Моментът \( S_n \) на появяване на \( n \)-тото събитие има плътност
    \begin{align*}
      f_{S_n}(t) = \lambda(t) e^{-\Lambda(t)} \frac {{(\Lambda(t))}^{n-1}} {(n-1)!}
    \end{align*}
  \end{enumerate}
\end{theorem}

\begin{note}
  Нехомогенния Поасонов процес ни позволява да моделираме, например, пристигането на посетители в магазин, което се променя драстично в течение на деня.
\end{note}

\printbibliography

\end{document}
