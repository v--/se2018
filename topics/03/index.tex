\documentclass{../../common/topic}

\TopicSetup
{
  number=3,
  title={Полиноми на една променлива. Теорема за деление с остатък. Най-голям общ делител на полиноми - тъждество на Безу и алгоритъм на Евклид. Зависимост между корени и коефициенти на полиноми (формули на Виет).},
  basedate={15 юни 2019},
  author={Янис Василев}
}

% Packages
\usepackage{polynom} % Polynomial long division

% Bibliography
\addbibresource{./references.bib}

\begin{document}

\maketitle

\section{Теория}

Освен посочените в конспекта български книги на Сидеров и Чакърян, полезни са и ранната българска книга \cite{Обрешков1962ВисшаАлгебра}, както и \cite{ГеновМиховскиМоллов1991Алгебра}. Някои твърдения и доказателства са заимствани от \cite{Knapp2016BasicAlgebra} и \cite{RoyachkiNotes}.

\subsection{Анотация}

Изложената анотацията е взета от \cite{Syllabus}.

\begin{enumerate}
  \item Полином с коефициенти над поле.
  \item Степен на полином.
  \item Корени на полиноми.
  \item Теорема за деление с остатък.
  \item Схема на Хорнер.
  \item Всеки идеал в \( F[x] \) е главен.
  \item Принцип за сравняване на коефициенти.
  \item Определение на най-голям общ делител на два полинома.
  \item Теорема за съществуване на най-голям общ делител на два полинома с коефициенти над поле.
  \item Изразяване на НОД чрез полиномите (тъждество на Безу).
  \item Алгоритъм на Евклид.
  \item Формули на Виет.
\end{enumerate}

\subsection{Основни понятия}

Нека \( F \) е фиксирано поле. За удобство ще означаваме с \( 0 \) и \( 1 \) съответно нулевият и единичният елемент на полето. Ще дефинираме полиноми като чисто алгебрични обекти вместо като функции. Причината за това е, че ако дефинираме полиноми като функции от вида\footnote{Трябва да отбележим, че някои автори, сред които и Никола Обрешков в \cite[1]{Обрешков1962ВисшаАлгебра}, предпочитат да означават с \( a_0 \) старшия коефициент, така че да имаме
\begin{equation*}
  p(x) = a_0 x^n + a_1 x^{n-1} + \cdots + a_{n-2} x^2 + a_{n-1} x + a_n
\end{equation*}
}
\begin{equation*}
  p(x) = a_n x^n + a_{n-1} x^{n-1} + \cdots + a_2 x^2 + a_1 x + a_0,
\end{equation*}
тогава в общия случай една и съща функция може да се дефинира по няколко различни начина.

Например, в полето \( F_2 \) с два функциите \( x \) и \( x^2 \) съвпадат. Това ни пречи да говорим без двусмислица за \enquote{степен}, \enquote{старши член}, \enquote{коефициент} на полином и подобни понятия.

\begin{definition}
  \textbf{Полином} \( p \) на една променлива над \( F \) наричаме редица
  \begin{equation*}
    p = (a_0, a_1, \ldots)
  \end{equation*}
  от елементи на \( F \), наречени коефициенти, само краен брой от които са различни от \( 0 \). Ако всички елементи на редицата са нули, наричаме полинома нулев и също както нулевия елемент на полето го бележим с \( 0 \).

  \textbf{Степен \( \deg(p) \)} на полинома \( p \) наричаме най-големият индекс, съответстващ на ненулев коефициент. Формално,
  \begin{equation*}
    \deg(p) \coloneqq \max \set[\Big]{ k = 0, 1, \ldots \given* a_k \neq 0 }.
  \end{equation*}

  По конвенция оставяме степента \( \deg(0) \) на нулевия полином да бъде неопределена (друг популярен вариант е да се положи \( \deg(0) \) да бъде \( -\infty \)).

  \textbf{Старшият коефициент} \( \op{LC}(p) \) на полинома \( p \) от степен \( n \) наричаме последната ненулева стойност в редицата от коефициенти и полагаме \( \op{LC}(p) \coloneqq 0 \).

  Полинома \( p \) наричаме \textbf{унитарен}, ако \( \op{LC}(p) = 1 \).
\end{definition}

Нека \( p = (a_0, a_1, \ldots) \) и \( q = (b_0, b_1, \ldots) \) са два полинома. Сума на \( p \) и \( q \) дефинираме покоординатно, т.е.
\begin{equation*}
  (p + q) = (a_0 + b_0, a_1 + b_1, \ldots),
\end{equation*}

а произведението им дефинираме като полинома \( pq = (c_0, c_1, \ldots) \), където
\begin{equation*}
  c_k = \sum_{i+j=k} a_i b_j.
\end{equation*}

\begin{remark}
  Това произведение се обобщава почти без изменение за т. нар. \enquote{групови алгебри} (\enquote{group algebra/ring}), където в някои случаи се нарича \enquote{конволюция}. За справка в.ж. \cite[пример B-1.1.]{Rotman2015AlgebraVol1}.
\end{remark}

Сумата на ненулеви полиноми \( p + q \) е полином, при това \( p + q \) или е нулевият полином, или \( \deg(p + q) \leq \max(\deg(p), \deg(q)) \). Произведението на ненулеви полиноми е ненулев полином, при това \( \deg(pq) = \deg p + \deg q \).

Полиноми със само един ненулев коефициент наричаме \textbf{мономи}.

Нека сега изберем символ, да речем \( X \), с който ще означаваме монома \( (0, 1, 0, 0, \ldots) \) (в \cref{rem:polynomials_as_syntax} ще дискутираме малко по-подробно ролята на \( X \)). Забелязваме, че от определението за умножение на полиноми, може да изразим коефициентите \( c_0, c_1, \ldots \) на \( X^2 = X \cdot X \) чрез коефициентите \( a_0, a_1, \ldots \) на \( X \) като
\begin{align*}
  c_0 &= a_0 \cdot a_0 = 0 \\
  c_1 &= a_0 \cdot a_1 + a_1 \cdot a_0 = 0 + 0 = 0 \\
  c_2 &= a_0 \cdot a_2 + a_1 \cdot a_1 + a_2 \cdot a_0 = 0 + 1 + 0 = 1 \\
  c_3 &= a_0 \cdot a_3 + a_1 \cdot a_2 + a_2 \cdot a_1 + a_0 \cdot a_3 = 0 + 0 + 0 + 0 = 0 \\
  c_4 &= \cdots = 0 \\
  c_5 &= \cdots = 0 \\
  \vdots
\end{align*}

По индукция така получаваме, че
\begin{equation*}
  X^k = (\underbrace{0, \ldots, 0}_{k \text{ пъти}}, 1, 0, 0, \ldots).
\end{equation*}

За удобство полагаме \( X^0 \coloneqq 1 \). Това ни позволява да записваме ненулевите полиноми \( p = (a_0, a_1, \ldots) \) от степен \( \deg(p) = n \) като линейна комбинация на мономи:
\begin{equation*}
  p(X) = \sum_{k=0}^n a_k X^k.
\end{equation*}

За променливата сме избрали главна буква, за да подчертаваме, че \( p(X) \) не е функция. Бележим с \( F[X] \) множеството на всички полиноми над \( F \) с променлива \( X \).

Относно въведените операции \( F[X] \) е комутативен пръстен с единица \( 1 \), тъй като
\begin{enumerate}
  \item \( F[X] \) наследява нулата си \( 0 \) и единицата си \( 1 \) от полето \( F \).
  \item Събирането на произволни полиноми наследява асоциативността и комутативността си директно от събирането в полето \( F \).
  \item Ако \( p(X) = \sum_{k=0}^n a_k X^k \), то \( -p(X) = \sum_{k=0}^n (-a_k) X^k \) е обратен на \( p(X) \) относно събиране.
  \item Произведението на ненулеви полиноми \( p = (a_0, a_1, \ldots) \), \( q = (b_0, b_1, \ldots) \) и \( r = (c_0, c_1, \ldots) \) е асоциативно, тъй като
  \begin{equation*}
    \sum_{k+m=n} \parens*{\sum_{i+j=k} a_i b_j } c_m
    =
    \sum_{i+j+m=n} a_i b_j c_m
    =
    \sum_{i+l=n} a_i \parens*{ \sum_{j+m=l} b_j c_m },
  \end{equation*}
  където всички индекси са неотрицателни цели числа.
  \item Произведението на ненулеви полиноми наследява комутативността си и дистрибутивността си относно събирането директно от полето \( F \).
\end{enumerate}

Това ни позволява да разглеждаме \( F \) като подпръстен на \( F[X] \) и да разглеждаме \( F[X] \) като алгебра над полето \( F \).

Нулевият полином и полиномите от степен \( 0 \) наричаме константи и чрез каноничната проекция \( \pi: (a_0, 0, \ldots) \mapsto a_0 \) ги отъждествяваме с първия им коефициент. Аналогично, каноничното влагане \( \iota: a_0 \mapsto (a_0, 0, \ldots) \) влага \( F \) във \( F[x] \).

Нека \( (F \mapsto F) \) е пръстенът от функции над \( F \) с операция композиция. Дефинираме хомоморфизма
\begin{align*}
  &\Phi: F[X] \mapsto (F \mapsto F) \\
  &\Phi\parens*{(a_0, a_1, \ldots, a_n, 0, 0, \ldots) } \coloneqq \parens*{ u \mapsto \sum_{k=0}^n a_k u^k },
\end{align*}
който на всеки полином съпоставя \textbf{полиномиална функция}. Както споменахме по-горе, този хомоморфизъм в общия случай не е инективен. Когато имаме предвид функцията \( x \mapsto \Phi(p)(x) \) вместо редицата от коефициенти \( p \), ще пишем \( p(x) \), като подобно означение ще използваме за стойността \( p(u) \) на функцията \( p(x) \) пресметната в точката \( u \).

\begin{remark}\label{rem:polynomials_as_syntax}
  В известен смисъл полиномите са синтактични обекти и от тази гледна точка \( X \) не е просто символ, защото играе ролята на променлива в смисъла на логиката и информатиката.

  Формално това може да се изрази с твърдението, че пръстенът \( F[X] \) е \enquote{свободната комутативна алгебра} над \( F \) породена от \( X \). Доказателство, както и съответните формализми, могат да бъдат намерени в \cite[теорема 1-3.25]{Rotman2015AlgebraVol1}.

  Грубо казано, такава алгебра е пръстен, чиито елементи можем да умножаваме с елементите на \( F \) (по-точно е линейно пространство с комутативна билинейна операция, която превръща линейното пространство в пръстен). Прилагателното \enquote{свободна} означава, че елементите на \( F[X] \), т.е. полиномите над \( F \) с променлива \( X \), са способни да опишат как взаимодействат елементите на всяка друга свободна комутативна алгебра над \( F \) породена от \( X \).
\end{remark}

\subsection{Делимост на полиноми}

\begin{theorem}[Делене с остатък]\label{thm:polynomial_division}
  Нека са дадени полиномите
  \begin{align*}
    p(X) = \sum_{k=0}^n a_k X^k &&\text{и}&& q(X) = \sum_{k=0}^m b_k X^k,
  \end{align*}
  където \( q(X) \neq 0 \). Тогава съществуват единствени полиноми \( s(X) \) и \( r(X) \), където \( r(X) = 0 \) или \( \deg(r) < m \), такива че
  \begin{equation*}
    p = sq + r.
  \end{equation*}
\end{theorem}
\begin{proof}
  \UniquenessSubProof Нека
  \begin{equation*}
    p = sq + r = \hat sq + \hat r.
  \end{equation*}

  Тогава
  \begin{equation*}
    0 = p - p = (s - \hat s) q + (r - \hat r)
  \end{equation*}
  и
  \begin{equation*}
    (s - \hat s) q = \hat r - r.
  \end{equation*}

  Тъй като \( q \neq 0 \), то \( s - \hat s = 0 \) тогава и само тогава, когато \( \hat r - r = 0 \). Ако сега допуснем, че \( \hat r \neq r \) (и следователно \( \hat s \neq s \)), получаваме, че
  \begin{equation*}
    \deg[(s - \hat s) q] = \deg(s - \hat s) + m > m.
  \end{equation*}

  Но по условие имаме
  \begin{equation*}
    \deg(\hat r - r) \leq \max(\deg \hat r, \deg r) < m.
  \end{equation*}

  Тъй като степента на полинома в двете страни на равенството трябва да бъде равна, получаваме противоречие от допускането, че \( \hat r \neq r \). Следователно \( r = \hat r \) и \( s = \hat s \).

  \ExistenceSubProof Ако \( n < m \), полагаме \( s(X) \coloneqq 0 \) и \( r(X) \coloneqq p(X) \).

  Остава случаят \( n \geq m \). Ще докажем теоремата с индукция по \( n \).

  При \( n = 0 \) полагаме \( s(X) \coloneqq b_0 / a_0 \) и \( r(X) \coloneqq 0 \). Да предположим, че теоремата е вярна за всички полиноми със степен по-малка от \( n \) и да положим
  \begin{equation*}
    g(X) \coloneqq \frac {a_n} {b_m} X^{n-m} q(X).
  \end{equation*}

  Тъй като \( \deg(p) = \deg(g) \) и \( \op{LC}(p) = \op{LC}(g) \), то \( \deg(p - g) < \deg(p) = n \) и индукционното предположение ни дава полиноми \( \hat s \) и \( \hat r \), такива че \( p - g = \hat s q + \hat r \) и \( \hat r = 0 \) или \( \deg(\hat r) < m \). Но ние имаме
  \begin{equation*}
    p(X)
    =
    g(X) + \hat s(X) q(X) + \hat r(X)
    =
    \parens*{ \frac {a_n} {b_m} X^{n-m} + \hat s(X) } q(X) + \hat r(X).
  \end{equation*}

  Полагаме
  \begin{align*}
    s(X) &\coloneqq \hat s(X) + \frac {a_n} {b_m} X^{n-m}, \\
    r(X) &\coloneqq \hat r(X).
  \end{align*}

  Очевидно \( \deg(r) = \deg(\hat r) < m \). С това и съществуването е доказано.
\end{proof}

\begin{definition}
  Казваме, че полиномът \( q \in F[X] \) \textbf{дели} \( p \in F[X] \) и че \( p \) е \textbf{кратен} на \( q \), ако съществува ненулев полином \( s \in F[X] \), такъв че \( p = sq \), т.е. ако алгоритъмът за делене с остатък дава нулев остатък.

  Множеството от всички полиноми, кратни на \( q \), образува идеал \( \braket q \) на пръстена \( F[X] \). \Cref{thm:polynomial_ideals_are_principal} ни казва, че всеки идеал на \( F[X] \) е от този вид.
\end{definition}

Полиномът \( q \) дели \( p \) тогава и само тогава, когато \( p \) да принадлежи на идеала \( \braket q \).

\begin{theorem}\label{thm:polynomial_ideals_are_principal}
  Всеки идеал в \( F[X] \) е главен.
\end{theorem}
\begin{proof}
  Нулевият идеал \( \braket 0 \) очевидно е главен. Нека \( I \) е ненулев идеал и нека \( q \in I \) е полином от минимална за \( I \) степен. Ще докажем, че идеалът \( \braket q \), породен от \( q \), съвпада с \( I \).

  Нека първо \( p \in \braket q \). Тогава съществува полином \( s(X) \), за който \( p = sq \). Но тъй като \( q \in I \), то \( p = sq \in I \). Тоест \( \braket q \subseteq I \).

  Нека сега \( p \in I \). Теоремата за делене с остатък ни дава полиноми \( s \) и \( r \) с \( r = 0 \) или \( \deg r < \deg q \), такива че \( p = qs + r \). Но понеже \( I \) е затворен относно събиране, имаме \( r = p - qs \in I \). Ако \( r \) е ненулев, то \( \deg r < \deg q \), което противоречи на минималността на \( q \). Значи \( r = 0 \) и \( p = qs \in \braket q \). Тоест \( I \subseteq \braket q \).

  Доказахме, че \( I = \braket q \). Понеже \( I \) беше произволен ненулев идеал, това означава, че всеки идеал на \( F[X] \) е главен.
\end{proof}

\subsection{Корени}

\begin{definition}
  \textbf{Корен} на полинома \( p(X) \) наричаме всяка стойност \( u \in F \), за която съответната функция се анулира, т.е. за която \( p(u) = 0 \).
\end{definition}

\begin{proposition}\label{thm:root_divisor}
  Полиномът \( (X - u) \) дели ненулевия полином \( p(X) \) тогава и само тогава, когато \( u \) е корен на \( p \).
\end{proposition}
\begin{proof}
  \SufficiencySubProof Ако \( (X - u) \) дели \( p(X) \), то \( p(X) \in \braket{X - u} \). Тъй като \( u \) е корен на полинома \( (X - u) \), той е корен и на всички полиноми от идеала \( \braket{X - u} \) и значи \( u \) е корен на \( p(X) \).

  \NecessitySubProof Нека \( u \) е корен на \( p(X) \).

  \Cref{thm:polynomial_division} ни дава полиноми \( q(X) \) и \( r(X) \), където или \( r(X) = 0 \), или \( \deg r < \deg b \), такива че
  \begin{equation*}
    p(X) = (X - u) q(X) + r(X).
  \end{equation*}

  Да допуснем, че полиномът \( r(X) \) е ненулев. Стойността на \( p(X) \) в \( u \) е
  \begin{equation*}
    0 = p(u) = (u - u) q(r) + r(u) = r(u),
  \end{equation*}
  следователно \( u \) е корен и на \( r(X) \). Но \( \deg r(X) < \deg (X - u) = 1 \), тоест \( r(X) \) е ненулев константен полином и \( r(X) \) не може да има нули. Полученото противоречие доказва твърдението.
\end{proof}

\begin{lemma}\label{thm:maximal_number_of_roots}
  Ненулев полином от степен \( n \) има най-много \( n \) корена, броейки кратностите.
\end{lemma}
\begin{proof}
  Ще използваме индукция по степента \( n \). В случая \( n = 0 \) имаме ненулев константен полином, а такъв полином не може да има корени, т.е. има най-много \( 0 \) корена.

  Да допуснем, че твърдението е вярно за \( n - 1 \). Нека \( p(X) \) е полином от степен \( n \) и нека \( u \) е негов корен. От \cref{thm:root_divisor} следва, че \( X - u \) дели \( p(X) \). Тогава съществува полином \( q(X) \) от степен \( n - 1 \), такъв че
  \begin{equation*}
    p(X) = (X - u) q(X).
  \end{equation*}

  Фиксираме елемент \( t \in F \), различен от \( u \) и от корените на \( q(X) \). Разглеждаме
  \begin{equation*}
    p(t) = (t - u) q(t).
  \end{equation*}

  Имаме \( (t - u) \neq 0 \) и \( q(t) \neq 0 \). Понеже \( F \) няма делители на нулата, произведението \( p(t) \) на ненулевите елемент \( (t - u) \) и \( q(t) \) също е ненулев елемент. Следователно единствените корени на \( p(X) \) са \( u \) и корените на \( q(X) \).

  По индукционно предположение, \( q(X) \) има най-много \( n - 1 \) корена, броейки кратностите. Следователно \( p(X) \) има най много \( (n - 1) + 1 = n \) корена.
\end{proof}

\begin{theorem}[Принцип за сравняване на коефициентите]\label{thm:coefficient_comparison_principle}
  Нека \( p(X) \) и \( q(X) \) са полиноми от степен \( n \) над \( F \) и нека \( u_0, \ldots, u_n \) са различни елементи на \( F \) (това изисква в полето има поне \( n+1 \) елемента). Ако \( p(u_i) = q(u_i) \) за всички \( i = 0, \ldots, n \), то полиномите \( p \) и \( q \) съвпадат.
\end{theorem}
\begin{proof}
  Разликата \( r(X) \coloneqq p(X) - q(X) \) е полином от степен най-много \( n \), който има \( n + 1 \) корена --- стойностите \( u_0, u_1 \ldots, u_n \). Според \cref{thm:maximal_number_of_roots}, това не е възможно за ненулев полином. Тоест \( r = 0 \) и \( p(X) = q(X) \).
\end{proof}

\subsection{Схема на Хорнер}

\begin{remark}
  Има определени разногласия относно кое именно е \enquote{правилото на Хорнер}.

  Самият Уилиам Хорнер в \cite{Horner1819SolvingNumericalEquations} описва метод за търсене на корени. Методът му в контекста на трансформиране на уравнения е описан в \cite[\S III.V.2]{Обрешков1962ВисшаАлгебра} като \enquote{правило на Хорнер}.

  От друга страна Доналд Кнут (чиято терминология е широко разпространена сред информатиците) в \cite[486]{Knuth1997ArtVol2} нарича \enquote{Horner's rule} (\enquote{правилото на Хорнер}) представянето
  \begin{equation}\label{eq:horners_rule/knuth}
    p(X) = \sum_{k=0}^n a_k X^k = a_0 + X (a_1 + \cdots + X(a_{n-1} + X a_n) + \cdots),
  \end{equation}
  разгледано от гледна точка на изчислителната сложност --- то изисква \( n \) умножения и \( n \) събирания, докато директното пресмятане на \( p(u) \) изисква \( {n(n+1)} / 2 \) умножения и \( n \) събирания.

  Както ще видим, \eqref{eq:horners_rule/knuth} като правило за пресмятане на стойностите на полином наистина е обосновано от рекурсията \eqref{eq:horners_rule/recursion}, която обаче е само част от метода на Хорнер.
\end{remark}

Нека сега разделим полинома
\begin{equation*}
  p(X) = \sum_{k=0}^n a_k X^k,
\end{equation*}
на \( X - u \):
\begin{equation*}
  p(X) = (X - u) q(X) + r(X).
\end{equation*}

Нека означим с \( b_0, b_1, \ldots, b_{n - 1} \) коефициентите на \( q(X) \). Според \cref{thm:polynomial_division} остатъкът \( r(X) \) е константа. Ще бележим тази константа с \( b_{-1} \). Тогава за неотрицателни цели \( k \) имаме
\begin{equation*}
  a_k = b_{k-1} - u b_k
\end{equation*}
и съответно
\begin{equation*}
  b_{k-1} = a_k + u b_k.
\end{equation*}

Това води до следната рекурсия:
\begin{equation}\label{eq:horners_rule/recursion}
  b_{-1} = a_0 + u b_0 = a_0 + u (a_1 + u b_1) = a_0 + u (a_1 + u (a_2 + b_2)) = \cdots = \sum_{k=0}^n a_k u^n = p(u),
\end{equation}
която при ръчно смятане е прието да се записва в табличен вид:
\begin{equation*}
  \begin{array}{c | c c c c c}
      & a_n     & a_{n-1} & \cdots & a_1 & a_0 \\
    u & b_{n-1} & b_{n-2} & \cdots & b_0 & b_{-1}
  \end{array}
\end{equation*}

Например, за \( p(X) = X^3 + 6 \) и \( u = -2 \) таблицата има вида
\begin{equation*}
  \begin{array}{c | c c c c}
       & 1 & 0                                  & 0                                  & 6 \\
    -2 & 1 & \underbrace{0 + (-2) \cdot 1}_{-2} & \underbrace{0 + (-2) \cdot (-2)}_4 & \underbrace{6 + (-2) \cdot 4}_{-2}
  \end{array}
\end{equation*}

Освен че така намираме стойността \( p(-2) \), ние получаваме и коефициентите на частното и остатъка на \( p(X) \) разделено на \( X + 2 \):
\begin{equation*}
  p(X) = (X + 2)(X^2 - 2X + 4) - 2.
\end{equation*}

Нещо повече, посредством последователно делене можем да намерим и резултата от субституцията на \( X \) с \( X + 2 \):
\begin{equation*}
  \begin{array}{c | c c c c}
       & 1        & 0         & 0         & 6         \\
    -2 & 1        & -2        & 4         & \fbox{-2} \\
    -2 & 1        & -4        & \fbox{12} &           \\
    -2 & 1        & \fbox{-6} &           &           \\
    -2 & \fbox{1} &           &           &           \\
  \end{array}
\end{equation*}

Така получаваме
\begin{equation*}
  p(X + 2) = (X + 2)^3 - 6(X + 2)^2 + 12(X + 2) - 2.
\end{equation*}

\subsection{Най-голям общ делител на полиноми}

\begin{definition}
  Нека фиксираме два полинома \( p \) и \( q \) над полето \( F \). Казваме, че \( d \) е \textbf{най-голям общ делител} (НОД) на \( p \) и \( q \), ако \( d \) дели \( p \) и \( q \) и ако всеки общ делител на \( p \) и \( q \) дели \( d \).

  Тъй като всички НОД на \( p \) и \( q \) се различават по умножение с ненулева константа, за определеност въвеждаме означението \( \gcd(p, q) \) за унитарния НОД.

  Казваме, че полиномите \( p \) и \( q \) са \textbf{взаимно прости}, ако \( \gcd(p, q) = 1 \).

  Оставяме НОД на два нулеви полинома да бъде неопределен.
\end{definition}

\begin{theorem}
  За всеки два полинома \( p \) и \( q \) над \( F \) съществува единствен с точност до умножение с ненулева константа най-малък общ делител.
\end{theorem}
\begin{proof}
  От \cref{thm:polynomial_ideals_are_principal} следва, че идеалът \( I = \braket p + \braket q \) е главен, т.е. съществува унитарен полином \( d \in I \), който го поражда.

  Тогава \( d \) е общ делител на \( p \) и \( q \). Но \( d \in I \), следователно съществуват полиноми \( u \) и \( v \), такива че
  \begin{equation*}
    u p + v q = d.
  \end{equation*}

  Тъй като \( \braket g \) съдържа \( p \) и \( q \) за всеки общ делител \( g \), имаме \( d = u p + v q \in \braket g \), т.е. \( g \) дели \( d \). Ако \( \deg g = \deg d \), то те се различават с ненулева константа.

  Тогава \( d \) е най-голям общ делител на \( p \) и \( q \).
\end{proof}

Като част от горното доказателство ние доказахме и следната
\begin{theorem}[Тъждество на Безу]
  За всеки два полинома \( p \) и \( q \) над поле съществуват полиноми \( u \) и \( v \), такива че
  \begin{equation*}
    u p + v q = \gcd(p, q).
  \end{equation*}
\end{theorem}

Ако \( p = 0 \) и \( q \neq 0 \), имаме \( \gcd(p, q) = p \) (и обратно). За ненулеви полиноми имаме явен алгоритъм за намиране на НОД.
\begin{theorem}[Алгоритъм на Евклид]
  Нека \( p \) и \( q \) са ненулеви полиноми над поле \( F \).

  Полагаме
  \begin{align*}
     f_{-1} \coloneqq p
     &&\text{и}&&
     f_0 \coloneqq q.
  \end{align*}

  Алгоритъм на Евклид (\( k \)-та стъпка; \( k \geq 1 \)): Деленето с остатък ни дава полиноми \( g_k \) и \( f_k \), такива че \( f_{k-2} = f_{k-1} g_k + f_k \).
  \begin{enumerate}
    \item Ако \( f_k = 0 \), то \( f_{k-1} \) е НОД на \( p \) и \( q \) и алгоритъмът приключва.
    \item Ако \( f_k \neq 0 \) и \( \deg f_k < \deg f_{k-1} \), преминаваме към стъпка \( k + 1 \).
  \end{enumerate}

  Твърдим, алгоритъмът приключва след краен брой стъпки и че резултатът му е НОД на \( p \) и \( q \).
\end{theorem}
\begin{proof}
  Броят стъпки е ограничен от степените на \( p \) и \( q \) тъй като на стъпка \( k \geq 1 \), ако \( f_k \) още не е \( 0 \), то \( \deg f_k < \deg f_{k-1} \) (възможно е обаче \( \deg f_0 > \deg f_{-1} \)).

  Нека \( m \geq 1 \) е последният индекс за който \( f_m \neq 0 \). С индукция по \( i = 0, \ldots, m + 1 \) ще докажем, че \( f_m \) дели \( f_{m-i} \):
  \begin{itemize}
    \item Очевидно \( f_m \) дели \( f_m \).
    \item Тъй като \( f_{m-1} = g_m f_m + f_{m+1} \) и по построение \( f_{m+1} = 0 \), оттук следва, че \( f_m \) дели \( f_{m-1} \).
    \item Нека допуснем, че \( f_m \) дели \( f_{m-(k-1)} \) и \( f_{m-(k-2)} \). Тогава
    \begin{equation*}
      f_{m-k} = g_{m-k+1} f_{m-k+1} + f_{m-k+2} = f_m \parens[\Big]{ g_{m-k+1} \frac {f_{m-k+1}} {f_m} + \frac {f_{m-k+2}} {f_m} }.
    \end{equation*}
  \end{itemize}

  Така доказахме, че \( f_m \) дели \( p = f_{-1} \) и \( q = f_0 \).

  Нека сега \( d \) е произволен общ делител на \( p \) и \( q \). Тогава
  \begin{equation*}
    f_1 = p - q g_1 = d \parens[\Big]{ \frac p d - g_1 \frac q d },
  \end{equation*}
  следователно \( d \) дели \( f_1 \). Със същото разсъждение и с индукция по \( i = 1, \ldots, m \) стигаме до извода, че \( d \) дели \( f_i \), в частност \( g \) дели \( f_m \). Следователно \( f_m \) е НОД на \( p \) и \( q \)
\end{proof}

\subsection{Формули на Виет}

\begin{theorem}[Формули на Виет]\label{thm:vietas_formulas}
  Нека над полето \( F \) е зададен унитарен полином
  \begin{equation*}
    p(X) = \sum_{k=0}^n a_k X^k
  \end{equation*}
  с положителна степен и нека всичките му корени \( u_1, \ldots, u_n \) (с евентуални повторения) са от \( F \).

  Тогава за \( k = 1, \ldots, n \) имаме следната връзка между коефициентите и корените на \( p \):
  \begin{equation*}
    a_{n-k} = a_n \cdot {(-1)}^k \sum_{1 \leq i_1 < \cdots < i_k \leq n} u_{i_1} \ldots u_{i_k}.
  \end{equation*}
\end{theorem}
\begin{proof}
  След като всички корени на \( p \) са във \( F \), то \( p \) се разлага на линейни множители над \( F[X] \), т.е.
  \begin{equation}\label{eq:thm:vietas_formulas/proof/assumption}
    p(X) = a_n (X - u_1) \cdots (X - u_n).
  \end{equation}

  Ще докажем теоремата с индукция по \( n = \deg p \). Базовият случай \( n = 1 \) е тривиален, тъй като тогава \( p(X) = (X - u_1) \) и \( a_0 = -u_1 \).

  Нека теоремата е вярна за всички полиноми от степен \( n - 1 \). Ще докажем теоремата за полинома \eqref{eq:thm:vietas_formulas/proof/assumption}.

  Индукционното предположение е изпълнено за
  \begin{equation*}
    q(X) \coloneqq (X - u_1) \cdots (X - u_{n-1}).
  \end{equation*}

  Нека означим коефициентите на \( q(X) \) с \( b_0, \ldots, b_{n-1} \). От правилото на Хорнер, за всички коефициенти на \( p(X) \) имаме
  \begin{equation*}
    a_k = b_{k-1} - u_n b_k,
  \end{equation*}
  където \( b_{-1} = 0 \).

  Тогава от индукционното предположение за \( k = 0 = (n - 1) - (n - 1) \) имаме
  \begin{equation*}
    a_{n-n}
    =
    a_0
    =
    b_{-1} - u_n b_0
    =
    u_n \prod_{k=0}^{n-1} u_k
    =
    \prod_{k=0}^n u_k
  \end{equation*}
  и, при \( 0 < k < n \), от индукционното предположение за \( k \) и за \( k + 1 \) имаме
  \begin{align*}
    a_{n-k}
    &=
    b_{n-k-1} - u_n b_{n-k}
    = \\ &=
    b_{(n-1)-k} - u_n b_{(n-1)-(k-1)}
    = \\ &=
    a_n \cdot {(-1)}^k \sum_{1 \leq i_1 < \cdots < i_k \leq n-1} u_{i_1} \ldots u_{i_k} - u_n \cdot a_n \cdot {(-1)}^{k-1} \sum_{1 \leq i_1 < \cdots < i_{k-1} \leq n-1} u_{i_1} \ldots u_{i_{k-1}}
    = \\ &=
    a_n \cdot {(-1)}^k \parens[\Bigg]{ \sum_{1 \leq i_1 < \cdots < i_k \leq n-1} u_{i_1} \ldots u_{i_k} + \sum_{1 \leq i_1 < \cdots < i_{k-1} \leq n-1} u_{i_1} \ldots u_{i_{k-1}} u_n }
    = \\ &=
    a_n \cdot {(-1)}^k \sum_{1 \leq i_1 < \cdots < i_k \leq n} u_{i_1} \ldots u_{i_k}.
  \end{align*}
\end{proof}

\section{Примерни задачи}

Условията на представените задачи са взети от \cite{PolynomialExercises}.

\subsection{Анотация}

\begin{enumerate}
  \item Намиране на НОД на два полинома - алгоритъм на Евклид, тъждество на Безу
  \item Прилагане на формулите на Виет за полином с числови коефициенти
\end{enumerate}

\subsection{Най-голям общ делител на полиноми}

\begin{problem}
  \hfill
  \begin{enumerate}
    \item Да се намери най-големият общ делител \( d(X) \) на полиномите
    \begin{align*}
      f(X) &\coloneqq X^3 + X^2 + X + 1, \\
      g(X) &\coloneqq X^2 - X + 2.
    \end{align*}

    \item Да се намерят полиноми \( u(X) \) и \( v(X) \), за които е изпълнено тъждеството на Безу
    \begin{equation*}
      d(X) = f(X) u(X) + g(X) v(X).
    \end{equation*}
  \end{enumerate}
\end{problem}
\begin{solution}
  \hfill
  \begin{enumerate}
    \item Делим \( f(X) \) на \( g(X) \):
    \begin{displaymath}
      \polylongdiv {X^3 + X^2 + X + 1} {X^2 - X + 2},
    \end{displaymath}

    Делим \( g(X) \) на \( f_1(X) \coloneqq X - 3 \):
    \begin{displaymath}
      \polylongdiv {X^2 - X + 2} {X - 3},
    \end{displaymath}

    Полиномът \( f_2(X) \coloneqq 8 \) дели \( f_1(X) \), следователно \( d(X) = f_2(X) = \gcd(f, g) = 8 \) и \( f(X) \) и \( g(X) \) са взаимно прости.

    \item Изразяваме остатъците от деленето при алгоритъма на Евклид:
    \begin{align*}
      f_1(X)
      &=
      f(X) - (X + 2) g(X),
      \\
      d(X)
      &=
      g(X) - (X + 2) f_1(X)
      = \\ &=
      g(X) - (X + 2) [f(X) - (X + 2) g(X)]
      = \\ &=
      (X + 2) f(X) + [{(X + 2)}^2 + 1] g(X)
      = \\ &=
      \boxed{(X + 2) f(X) + (X^2 + 4X + 5) g(X)}.
    \end{align*}
  \end{enumerate}
\end{solution}

\subsection{Формули на Виет}

\begin{problem}
  За кои стойности на параметъра \( p \in \BbbR \) корените \( u_1, \ldots, u_4 \) на полинома
  \begin{equation*}
    f(X) = X^4 - 8X^3 + 22X^2 + pX + 16
  \end{equation*}
  изпълняват равенството \( u_1 + u_2 + u_3 = u_4 \)?
\end{problem}
\begin{solution}
  Заместваме \( u_4 = u_1 + u_2 + u_3 \) във формулите на Виет:
  \begin{align*}
    8 &= (u_1 + u_2 + u_3) + u_4 = 2u_4 = 8
    \\&\implies
    u_4 = 4,
    \\ \\
    22 &= u_1 u_2 + u_1 u_3 + u_1 u_4 + u_2 u_3 + u_2 u_4 + u_3 u_4 = \\ &= u_1 u_2 + u_1 u_3 + u_2 u_3 + (u_1 + u_2 + u_3) u_4
    \\&\implies
    u_1 u_2 + u_1 u_3 + u_2 u_3 = 6,
    \\ \\
    -p &= u_1 u_2 u_3 + u_1 u_2 u_4 + u_1 u_3 u_4 + u_2 u_3 u_4 = \\ &= u_1 u_2 u_3 + (u_1 u_2 + u_1 u_3 + u_2 u_3) u_4 \\&\implies
    u_1 u_2 u_3 = -p - 24,
    \\ \\
    16 &= (u_1 u_2 u_3) u_4
    \\&\implies
    (-p - 24) 4 = 16 \implies p = -28.
  \end{align*}
\end{solution}

\begin{sloppypar}
  \printbibliography
\end{sloppypar}

\end{document}
